% Chapter 1

\chapter{Introduction} % Main chapter title

\label{Chapter1} % For referencing the chapter elsewhere, use \ref{Chapter1}

%----------------------------------------------------------------------------------------

% Define some commands to keep the formatting separated from the content
\newcommand{\keyword}[1]{\textbf{#1}}
\newcommand{\tabhead}[1]{\textbf{#1}}
\newcommand{\code}[1]{\texttt{#1}}
\newcommand{\file}[1]{\texttt{\bfseries#1}}
\newcommand{\option}[1]{\texttt{\itshape#1}}

%----------------------------------------------------------------------------------------

\section{Project description}
The process of urbanisation has been a double edged sword whereby many a people have benefited from it while a lot more have not. The issue of informal settlements is one that covers areas such socio-economic, governance, climate, politics, healthcare, and resource management to name a few.\\
One of the few approaches to tackle these issues is through good modelling, and understanding the growth of such informal settlements.\\
This project will employ a cellular automata technique, specifically John Conway's 'Game of Life' to model informal settlement growth in South Africa.
\section{Problem description and background}
\label{sec:prob}
From the 1950s to the early 2000s the rate of urbanisation has increased 20\%, however the amount of people living in inadequate housing, informal settlements, and slums globally is still about 1 billion.\cite{un}\\
According to the United Nations (UN) the definition of an informal settlement is a dwelling with a lack of security, sanitation, water, living area, and housing durability.\cite{un1} \\
The UN also has a list of 17 Sustainable Development Goals of which number 11 is to create 'Sustainable Cities and Communities'. These goals are such that if even a few can be achieved the other will become easier to achieve as well.
Discuss Informal settlement framework(what, where, when, how of informal settlements)\\\\
Cellular Automata (abbreviated as CA) is a discrete computational model which is studied in automata theory. The basic components of such a model is a grid which contains cells. Each cell can have a finite number of states it can take on. An initial state at time ($t = 0$) is assigned to the grid as a whole. For each time interval thereafter the cells change their states according to a predefined set of rules.\cite{ca}\\\\
Conway's \textit{Game of Life} also known as \textit{Life} was created by the British Mathematician John H Conway in the 1970s and first appeared in the \textit{Scientific American} magazine.\cite{conway}\\
The states the cells in Life can take on are either alive or dead. The rules that govern the states of Life are as follows:
\begin{enumerate}
\item Due to under-population a cell will die if it has less than 2 neighbours\footnote{These are cell which are alive and are located around the current cell}.
\item If a cell has 3 or 2 neighbours it will remain alive in the next cycle.
\item If a cell has more than 3 neighbours it will die.
\item If a dead cell is surrounded by 3 alive cells it will become alive in the next cycle.
\end{enumerate}
Using such a set of rules this project will embark on creating a model that will accurately predict informal settlement growth in South Africa.\\
Some limitations have been noted in using CA to model urban growth. These include creating trade-offs between flexibility and simplicity for the transitional rules. At the same time other opportunities in CA models also present themselves for study such as calibration, stochastic components, and cell types.\cite{ca1}
\section{Aims and objectives}
\subsection{Aims}
\begin{itemize}
\item Create a sound mathematical and statistical model
\item Apply model to Life to predict growth of informal settlements
\end{itemize}
\subsection{Objectives}
\begin{itemize}
\item Conduct the relevant literature reviews
\item Get access to relevant maps needed regarding informal settlements
\item If above step fails, create own maps
\item Create an application that allows a map to be added and a grid to be placed for simulating Life.
\item Create a mathematical and statistical model to provide an initial state
\item Apply initial state to the map
\item Iterate and monitor growth output
\item Calculate accuracy of model by comparing maps from different time periods
\end{itemize}