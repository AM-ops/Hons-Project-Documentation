% Chapter 5
\chapter{Reflection \& Conclusion} % Main chapter title
\label{Chapter5} % For referencing the chapter elsewhere, use \ref{Chapter5}
%6.1	What did you learn while completing this project?  Refer to the decisions that you took, and evaluate them.  Analyse the strong and weak points of the product and the process that you followed.  What would you do differently ?
The aims and objectives of this research study have all been achieved and have went slightly beyond the author's initial expectations.
\begin{enumerate}
\item A sound mathematical and statistical model has been created, and
\item it has been been successfully applied in the South African context to simulate the growth of the informal settlement called \textit{Melusi} using Cellular Automata and \textit{Life}.
\end{enumerate}

This research study was completed in favourable time and all major milestones were covered within the time specifications given by the Department.
\\\\
The author has learnt a great deal of expert knowledge while carrying out this research. The topics that played a vital role to the success were the following but not limited to GIS, Image Processing, and further Python development and modelling.

The main decision that made this research head in a totally different direction was the use of images as the base for the modelling process. However this could have been done differently by utilising the Python APIs available for both ArcGIS and QGIS. These software suites in general could have also been helpful in assisting with a number of tasks, however due to constraints in time and other responsibilities learning this technology was stunted. 

The use of APIs or the lack thereof was not entirely a suboptimal decision due to the fact that this simple (yet not trivial) approach of using Imagery and Data Structures to achieve a adequate model can open up this research space in the South African context and allow anyone with the tools and methodologies to tackle complex problems. The image processing procedures can in the future also be automated thereby reducing the time to create models.

The wait period for closed sourced data access was an unexpected hurdle which had to be overcome utilising the time for further research and insights into different approaches that could be taken to tackle this research study.

The effort from public and private sector should be to bolster access to data which can drastically improve modelling capabilities in South Africa.

The following aspects can be delved into for further research in the South African context:
\begin{itemize}
\item Utilise a great more deal of variables for the modelling.
\item Calibration of the model(s).
\item Usage of Fuzzy Logic, or Stochastic Process.
\item Adding Remote Sensing Science, Artificial Intelligence, and Agent Modelling to the dimensions of the research.
\item Creating custom software specific to the needs of Urban Development in South Africa.
\item Integrate the Cellular Automata model with other models.
\item Examine the different Urban ecological, Social physical, Neoclassical, Behavioural, and Systems approaches to view Cellular Automata modelling.
\end{itemize}
In closing, this research study was as insightful as it was necessary for good modelling to exist through the usage of good tools.