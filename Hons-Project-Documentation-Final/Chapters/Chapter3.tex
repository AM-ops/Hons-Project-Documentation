% Chapter 3

\chapter{Artefact Design \& Development} % Main chapter title
\label{Chapter3} % For referencing the chapter elsewhere, use \ref{Chapter3}
\section{Description of Artefact}
The nature of the research that is conducted in this study leads to a contrasting artefact that is delivered. The artefact is a culmination of image processing, programming, GIS data handling, and problem solving. This is made obvious as this chapter progresses further. In essence the artefact is the steps and procedures followed to come up with the final CA model.
\section{Life Cycle Methodology}
For the entire development of the Artefact a combination of three methodologies was utilised. These included:
\begin{itemize}
\item Waterfall
\item Agile
\item Scrum
\end{itemize}
The rationale for utilising this combination was due to the following factors:
\begin{itemize}
\item The emphasis on self-management for the development of the Artefact
\item Incremental changes made in short iterations
\item Segmenting problems into "chunks" and thereafter working on them in an iterative manner
\end{itemize}
\section{Data acquisition}
\subsection{Closed Sourced Data}
After the Literature Review was conducted the search for data began. A company by the name GeoTerraImage (Pty) Ltd\footnote{\url{https://geoterraimage.com/}} which is located in Pretoria, South Africa was contacted and a data request was filed. The data received was for an Informal settlement named \textit{"Melusi"}. The author is grateful for the data provided. The dataset received contained three main points of interest to this research:
\begin{itemize}
\item The outline or boundary of the \textit{"Melusi"} area.
\item The Land Usage in the form of housing for the year 2010.
\item The Land Usage in the form of housing for the year 2020.
\end{itemize}
\subsection{Open Sourced Data}
\section{Data Exploration and Preprocessing}
\subsection{Usage of Geographic Information Systems}
The use of Geographic Information Systems (GIS) were not greatly successful in this research. The factors limiting the usage were; the lack of packages, libraries, add-ons, or plug-ins that function "out-of-the-box", the ones that do work tend to fulfil niched application or specific problems.

Additional issues encountered were as follows:
\begin{itemize}
\item Incompatibility of Operating Systems
\item Data formats were incompatible with packages
\end{itemize}
The following GIS tools were utilised in the early stages of the research:
\begin{itemize}
\item ESRI's ArcGIS (Proprietary commercial software)
\item QGIS (Open-source software)
\end{itemize}

The following packages/plugins/libraries were also utilised in the early stages of the research:
\begin{itemize}
\item MOLUSCE - A plugin for QGIS for Land Use Change Evaluation\footnote{\url{https://wiki.gis-lab.info/w/Landscape_change_analysis_with_MOLUSCE_-_methods_and_algorithms}}
\item TerraME - A Multiparadigm Modeling Toolkit\footnote{\url{http://www.terrame.org/doku.php}}
\item GeoSOS (Geographic Simulation \& Optimization System) - A standalone program or as an add-on for ArcGIS\footnote{\url{https://www.geosimulation.cn/GeoSOS/}}
\end{itemize}
\subsection{GIS Data handling}
The 