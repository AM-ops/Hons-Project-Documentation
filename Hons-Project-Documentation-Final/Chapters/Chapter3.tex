% Chapter 3

\chapter{Development of Artefact} % Main chapter title
\label{Chapter3} % For referencing the chapter elsewhere, use \ref{Chapter3}
\section{Description of Artefact}
The nature of the research that is conducted in this study leads to a contrasting artefact that is delivered. The artefact is a culmination of image processing, programming, GIS data handling, and problem solving. This is made obvious as this chapter progresses further. In essence the artefact is the steps and procedures followed to come up with the final CA model.
\section{Life cycle}

\section{Data Exploration and Preprocessing}
\subsection{Usage of Geographic Information Systems}
The use of Geographic Information Systems (GIS) were not greatly successful in this research. The factors limiting the usage were; lack of packages or libraries that function "out-of-the-box", the ones that do work tend to fulfil niched application or specific problems.

Additional issues encountered were as follows:
\begin{itemize}
\item Incompatibility of Operating Systems with the packages/libraries/addon/plugins
\item Data formats were incompatible with packages/libraries/addon/plugins
\end{itemize}
The following GIS tools were utilised in the early stages of the research:
\begin{itemize}
\item ESRI's ArcGIS (Proprietary commercial software)
\item QGIS (Open-source software)
\end{itemize}

The following packages/plugins/libraries were also utilised in the early stages of the research:
\begin{itemize}
\item MOLUSCE - A plugin for QGIS for Land Use Change Evaluation\footnote{\url{https://wiki.gis-lab.info/w/Landscape_change_analysis_with_MOLUSCE_-_methods_and_algorithms}}
\item TerraME - A Multiparadigm Modeling Toolkit\footnote{\url{http://www.terrame.org/doku.php}}
\item GeoSOS (Geographic Simulation \& Optimization System) - A standalone program or as an add-on for ArcGIS\footnote{\url{https://www.geosimulation.cn/GeoSOS/}}
\end{itemize}
\subsection{GIS Data handling}
The 